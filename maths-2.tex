\documentclass[a4paper]{article}
\usepackage{color}
\usepackage{enumerate}
\usepackage{multirow}
\begin{document}

% this is to display answers
\newcommand{\answer}[1]{
  \textcolor{red}{#1}
}
% build with pdflatex "\def\showanswers{1}\documentclass[a4paper]{article}
\usepackage{color}
\usepackage{enumerate}
\usepackage{multirow}
\begin{document}

\input{
  |"/bin/echo $ANSWERS" %$
}

% this is to display answers
\newcommand{\answer}[1]{
  \textcolor{red}{#1}
}
% build with pdflatex "\def\showanswers{1}\documentclass[a4paper]{article}
\usepackage{color}
\usepackage{enumerate}
\usepackage{multirow}
\begin{document}

\input{
  |"/bin/echo $ANSWERS" %$
}

% this is to display answers
\newcommand{\answer}[1]{
  \textcolor{red}{#1}
}
% build with pdflatex "\def\showanswers{1}\documentclass[a4paper]{article}
\usepackage{color}
\usepackage{enumerate}
\usepackage{multirow}
\begin{document}

\input{
  |"/bin/echo $ANSWERS" %$
}

% this is to display answers
\newcommand{\answer}[1]{
  \textcolor{red}{#1}
}
% build with pdflatex "\def\showanswers{1}\input{maths-2.tex}" to show answers
\ifdefined\showanswers
\else
  \renewcommand{\answer}[1]{}
\fi


\title{Numeracy Evaluation 2 \answer{with answers}}
\author{Organic Trader Pty Ltd}

\maketitle
\noindent This evaluation is designed to give us an indication of your
maths abilities. Please read carefully and answer the following questions as best you
can. If you don't understand something please ask.

\hspace{5mm}

\noindent Name: \rule{3cm}{0.2pt}

\hspace{5mm}

\noindent Date: \rule{3cm}{0.2pt}



\section{No calculator to be used}

Please answer the following questions without the use of a calculator. You may use the space provided for doing your calculations.

\begin{enumerate}
\item \begin{math}  37 + 14 = \end{math} \answer{51}
\item \begin{math}  13 \times 7 = \end{math} \answer{91}
\item   20\%  of 80 = \answer{16}
\item \begin{math}  84 - 19 = \end{math} \answer{65}
\item \begin{math}  44 \times 1.5 = \end{math} \answer{66}
\item \begin{math}  \frac{1}{3} + \frac{1}{3} = \end{math} \answer{\begin{math}\frac{2}{3}\end{math}}
\item \begin{math}  \frac{1}{3} + \frac{2}{3} = \end{math} \answer{1}
\item \begin{math}  \frac{1}{4} + \frac{1}{2} = \end{math} \answer{\begin{math}\frac{3}{4}\end{math}}
\item \begin{math}  \frac{1}{2} \times \frac{1}{4} = \end{math} \answer{\begin{math}\frac{1}{8}\end{math}}
\item \begin{math}  \frac{1}{4} \times \frac{1}{4} = \end{math} \answer{\begin{math}\frac{1}{16}\end{math}}
\item \begin{math}  \frac{1}{3} \times \frac{1}{2} = \end{math} \answer{\begin{math}\frac{1}{6}\end{math}}
\item What is a half of a quarter? \answer{\begin{math}\frac{1}{8}\end{math}}
\item What is a half plus a quarter? \answer{\begin{math}\frac{3}{4}\end{math}}
\item Round the following numbers to two decimal places: 
  \begin{enumerate}
    \item 51.3424 = \answer{51.34}
    \item 754.6395 = \answer{754.64}
    \item 633.77872 = \answer{633.78}
    \item 0.5666256 = \answer{0.57}
    \item 21.999 = \answer{22.00}
    \item 5013.3333 = \answer{5013.33}
    \item 3.114999 = \answer{3.11}
    \item 17.10059 = \answer{17.10}
  \end{enumerate}
\end{enumerate}
\newpage
\section{Calculator may be used}
Please answer the following questions. You may use a calculator to help you answer the questions in this section.
\begin{enumerate}\addtolength{\itemsep}{3\baselineskip}

\item \begin{math} 27284 + 36723 = \end{math} \answer{64007}
\item \begin{math} 17.71 \times 12.45 = \end{math} \answer{220.4895}
\item \begin{math} 272 - 89 = \end{math} \answer{183}
\item \begin{math} 6283.87 \div 19 = \end{math} \answer{330.73}
\item \begin{math} 2331 + 39 \times 17 = \end{math} \answer{2994}
\item One of the products we sell, Pur Peppermint Gum comes in `12 packs'. Those are then packed in larger cartons called a mastercase.
  There are fourteen 12 packs in a mastercase.
  

  A customer orders 204 packs of Pur Peppermint Gum, how many master cases and 12 packs should we ship to them?
  
  
  \answer{Answer: 1 mastercase  plus three 12 packs }
\item In Australia, GST is set at 10\%. If a product is priced at
  \$100.00 including GST, what is the price excluding GST?

  \answer{Answer: \$90.91 }

\item Hidden Garden make a range of gluten free cookies. One of their products comes in a carton of 10 units, with a unit price of \$5.42.
  If we give the store a 15\% discount, how much will a carton cost?

  \answer{Answer: \$46.07}
  

\end{enumerate}
\end{document}

%%% Local Variables:
%%% mode: latex
%%% TeX-master: t
%%% End:
" to show answers
\ifdefined\showanswers
\else
  \renewcommand{\answer}[1]{}
\fi


\title{Numeracy Evaluation 2 \answer{with answers}}
\author{Organic Trader Pty Ltd}

\maketitle
\noindent This evaluation is designed to give us an indication of your
maths abilities. Please read carefully and answer the following questions as best you
can. If you don't understand something please ask.

\hspace{5mm}

\noindent Name: \rule{3cm}{0.2pt}

\hspace{5mm}

\noindent Date: \rule{3cm}{0.2pt}



\section{No calculator to be used}

Please answer the following questions without the use of a calculator. You may use the space provided for doing your calculations.

\begin{enumerate}
\item \begin{math}  37 + 14 = \end{math} \answer{51}
\item \begin{math}  13 \times 7 = \end{math} \answer{91}
\item   20\%  of 80 = \answer{16}
\item \begin{math}  84 - 19 = \end{math} \answer{65}
\item \begin{math}  44 \times 1.5 = \end{math} \answer{66}
\item \begin{math}  \frac{1}{3} + \frac{1}{3} = \end{math} \answer{\begin{math}\frac{2}{3}\end{math}}
\item \begin{math}  \frac{1}{3} + \frac{2}{3} = \end{math} \answer{1}
\item \begin{math}  \frac{1}{4} + \frac{1}{2} = \end{math} \answer{\begin{math}\frac{3}{4}\end{math}}
\item \begin{math}  \frac{1}{2} \times \frac{1}{4} = \end{math} \answer{\begin{math}\frac{1}{8}\end{math}}
\item \begin{math}  \frac{1}{4} \times \frac{1}{4} = \end{math} \answer{\begin{math}\frac{1}{16}\end{math}}
\item \begin{math}  \frac{1}{3} \times \frac{1}{2} = \end{math} \answer{\begin{math}\frac{1}{6}\end{math}}
\item What is a half of a quarter? \answer{\begin{math}\frac{1}{8}\end{math}}
\item What is a half plus a quarter? \answer{\begin{math}\frac{3}{4}\end{math}}
\item Round the following numbers to two decimal places: 
  \begin{enumerate}
    \item 51.3424 = \answer{51.34}
    \item 754.6395 = \answer{754.64}
    \item 633.77872 = \answer{633.78}
    \item 0.5666256 = \answer{0.57}
    \item 21.999 = \answer{22.00}
    \item 5013.3333 = \answer{5013.33}
    \item 3.114999 = \answer{3.11}
    \item 17.10059 = \answer{17.10}
  \end{enumerate}
\end{enumerate}
\newpage
\section{Calculator may be used}
Please answer the following questions. You may use a calculator to help you answer the questions in this section.
\begin{enumerate}\addtolength{\itemsep}{3\baselineskip}

\item \begin{math} 27284 + 36723 = \end{math} \answer{64007}
\item \begin{math} 17.71 \times 12.45 = \end{math} \answer{220.4895}
\item \begin{math} 272 - 89 = \end{math} \answer{183}
\item \begin{math} 6283.87 \div 19 = \end{math} \answer{330.73}
\item \begin{math} 2331 + 39 \times 17 = \end{math} \answer{2994}
\item One of the products we sell, Pur Peppermint Gum comes in `12 packs'. Those are then packed in larger cartons called a mastercase.
  There are fourteen 12 packs in a mastercase.
  

  A customer orders 204 packs of Pur Peppermint Gum, how many master cases and 12 packs should we ship to them?
  
  
  \answer{Answer: 1 mastercase  plus three 12 packs }
\item In Australia, GST is set at 10\%. If a product is priced at
  \$100.00 including GST, what is the price excluding GST?

  \answer{Answer: \$90.91 }

\item Hidden Garden make a range of gluten free cookies. One of their products comes in a carton of 10 units, with a unit price of \$5.42.
  If we give the store a 15\% discount, how much will a carton cost?

  \answer{Answer: \$46.07}
  

\end{enumerate}
\end{document}

%%% Local Variables:
%%% mode: latex
%%% TeX-master: t
%%% End:
" to show answers
\ifdefined\showanswers
\else
  \renewcommand{\answer}[1]{}
\fi


\title{Numeracy Evaluation 2 \answer{with answers}}
\author{Organic Trader Pty Ltd}

\maketitle
\noindent This evaluation is designed to give us an indication of your
maths abilities. Please read carefully and answer the following questions as best you
can. If you don't understand something please ask.

\hspace{5mm}

\noindent Name: \rule{3cm}{0.2pt}

\hspace{5mm}

\noindent Date: \rule{3cm}{0.2pt}



\section{No calculator to be used}

Please answer the following questions without the use of a calculator. You may use the space provided for doing your calculations.

\begin{enumerate}
\item \begin{math}  37 + 14 = \end{math} \answer{51}
\item \begin{math}  13 \times 7 = \end{math} \answer{91}
\item   20\%  of 80 = \answer{16}
\item \begin{math}  84 - 19 = \end{math} \answer{65}
\item \begin{math}  44 \times 1.5 = \end{math} \answer{66}
\item \begin{math}  \frac{1}{3} + \frac{1}{3} = \end{math} \answer{\begin{math}\frac{2}{3}\end{math}}
\item \begin{math}  \frac{1}{3} + \frac{2}{3} = \end{math} \answer{1}
\item \begin{math}  \frac{1}{4} + \frac{1}{2} = \end{math} \answer{\begin{math}\frac{3}{4}\end{math}}
\item \begin{math}  \frac{1}{2} \times \frac{1}{4} = \end{math} \answer{\begin{math}\frac{1}{8}\end{math}}
\item \begin{math}  \frac{1}{4} \times \frac{1}{4} = \end{math} \answer{\begin{math}\frac{1}{16}\end{math}}
\item \begin{math}  \frac{1}{3} \times \frac{1}{2} = \end{math} \answer{\begin{math}\frac{1}{6}\end{math}}
\item What is a half of a quarter? \answer{\begin{math}\frac{1}{8}\end{math}}
\item What is a half plus a quarter? \answer{\begin{math}\frac{3}{4}\end{math}}
\item Round the following numbers to two decimal places: 
  \begin{enumerate}
    \item 51.3424 = \answer{51.34}
    \item 754.6395 = \answer{754.64}
    \item 633.77872 = \answer{633.78}
    \item 0.5666256 = \answer{0.57}
    \item 21.999 = \answer{22.00}
    \item 5013.3333 = \answer{5013.33}
    \item 3.114999 = \answer{3.11}
    \item 17.10059 = \answer{17.10}
  \end{enumerate}
\end{enumerate}
\newpage
\section{Calculator may be used}
Please answer the following questions. You may use a calculator to help you answer the questions in this section.
\begin{enumerate}\addtolength{\itemsep}{3\baselineskip}

\item \begin{math} 27284 + 36723 = \end{math} \answer{64007}
\item \begin{math} 17.71 \times 12.45 = \end{math} \answer{220.4895}
\item \begin{math} 272 - 89 = \end{math} \answer{183}
\item \begin{math} 6283.87 \div 19 = \end{math} \answer{330.73}
\item \begin{math} 2331 + 39 \times 17 = \end{math} \answer{2994}
\item One of the products we sell, Pur Peppermint Gum comes in `12 packs'. Those are then packed in larger cartons called a mastercase.
  There are fourteen 12 packs in a mastercase.
  

  A customer orders 204 packs of Pur Peppermint Gum, how many master cases and 12 packs should we ship to them?
  
  
  \answer{Answer: 1 mastercase  plus three 12 packs }
\item In Australia, GST is set at 10\%. If a product is priced at
  \$100.00 including GST, what is the price excluding GST?

  \answer{Answer: \$90.91 }

\item Hidden Garden make a range of gluten free cookies. One of their products comes in a carton of 10 units, with a unit price of \$5.42.
  If we give the store a 15\% discount, how much will a carton cost?

  \answer{Answer: \$46.07}
  

\end{enumerate}
\end{document}

%%% Local Variables:
%%% mode: latex
%%% TeX-master: t
%%% End:
" to show answers
\ifdefined\showanswers
\else
  \renewcommand{\answer}[1]{}
\fi


\title{Numeracy Evaluation 2 \answer{with answers}}
\author{Organic Trader Pty Ltd}

\maketitle
\noindent This evaluation is designed to give us an indication of your
maths abilities. Please read carefully and answer the following questions as best you
can. If you don't understand something please ask.

\hspace{5mm}

\noindent Name: \rule{3cm}{0.2pt}

\hspace{5mm}

\noindent Date: \rule{3cm}{0.2pt}



\section{No calculator to be used}

Please answer the following questions without the use of a calculator. You may use the space provided for doing your calculations.

\begin{enumerate}
\item \begin{math}  37 + 14 = \end{math} \answer{51}
\item \begin{math}  13 \times 7 = \end{math} \answer{91}
\item   20\%  of 80 = \answer{16}
\item \begin{math}  84 - 19 = \end{math} \answer{65}
\item \begin{math}  44 \times 1.5 = \end{math} \answer{66}
\item \begin{math}  \frac{1}{3} + \frac{1}{3} = \end{math} \answer{\begin{math}\frac{2}{3}\end{math}}
\item \begin{math}  \frac{1}{3} + \frac{2}{3} = \end{math} \answer{1}
\item \begin{math}  \frac{1}{4} + \frac{1}{2} = \end{math} \answer{\begin{math}\frac{3}{4}\end{math}}
\item \begin{math}  \frac{1}{2} \times \frac{1}{4} = \end{math} \answer{\begin{math}\frac{1}{8}\end{math}}
\item \begin{math}  \frac{1}{4} \times \frac{1}{4} = \end{math} \answer{\begin{math}\frac{1}{16}\end{math}}
\item \begin{math}  \frac{1}{3} \times \frac{1}{2} = \end{math} \answer{\begin{math}\frac{1}{6}\end{math}}
\item What is a half of a quarter? \answer{\begin{math}\frac{1}{8}\end{math}}
\item What is a half plus a quarter? \answer{\begin{math}\frac{3}{4}\end{math}}
\item Round the following numbers to two decimal places: 
  \begin{enumerate}
    \item 51.3424 = \answer{51.34}
    \item 754.6395 = \answer{754.64}
    \item 633.77872 = \answer{633.78}
    \item 0.5666256 = \answer{0.57}
    \item 21.999 = \answer{22.00}
    \item 5013.3333 = \answer{5013.33}
    \item 3.114999 = \answer{3.11}
    \item 17.10059 = \answer{17.10}
  \end{enumerate}
\end{enumerate}
\newpage
\section{Calculator to be used}
Please answer the following questions. You may use a calculator to help you answer the questions in this section.
\begin{enumerate}\addtolength{\itemsep}{3\baselineskip}

\item \begin{math} 27284 + 36723 = \end{math} \answer{64007}
\item \begin{math} 17.71 \times 12.45 = \end{math} \answer{220.4895}
\item \begin{math} 272 - 89 = \end{math} \answer{183}
\item \begin{math} 6283.87 \div 19 = \end{math} \answer{330.73}
\item \begin{math} 2331 + 39 \times 17 = \end{math} \answer{2994}
\item One of our products, Cocolo Milk Hazelnut Chocolate comes in boxes of 18 bars called inners. Those inners are then packed into larger cartons called outers. There are 10 inners to the outer. 

A customer orders 252 bars of chocolate. How many inners and outers will we ship to them?

\answer{Answer: 1 outer plus 4 inners }

\item Another product comes as 24 units per carton. 

A customer orders 72 units. How many cartons should we ship to them?

\answer{Answer: 3 cartons }


\item In Australia, GST is set at 10\%. If a product is priced at
  \$100.00 including GST, what is the price excluding GST?
  \answer{Answer: \$90.91 }



\end{enumerate}
\end{document}

%%% Local Variables:
%%% mode: latex
%%% TeX-master: t
%%% End:
